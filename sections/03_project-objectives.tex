\subsection{Reliable and Accurate Backscatter Cancellation}
The system must be able to accurately and reliably pinpoint the location of backscatter particles as well as map out the shape of the outline, enabling the perfect segmentation of each particle for elimination. Although previous work on this project \cite{katieshepherdMachineVisionBased2023} outlines the backscatter detection and elimination system revolving around a simple blob detection algorithm, this project will focus on a Canny-based approach with the snake method optimisation. Harnessing the simplicity of a CPU-based machine, a high-level programming language such as Python, and heavily optimised libraries from OpenCV for rapid machine vision application development, an RPi 4 or 5 with an RPi Global Shutter camera will drive a DLP projector, utilising the particle centre location and outline mapping to project holes in the light beam in addition to the implementation of position calibration, which is crucial to minimise the parallax effect, the displacement of the target from the perceived position in the video capture, ultimately ensuring accurate backscatter elimination.

\subsection{Architectures and Methodologies for Real-Time}
The work in \cite{katieshepherdMachineVisionBased2023} discusses the hindrance to the successful operation of the blob detection-based backscatter elimination software due to the effects of the Linux operating system (OS): "Linux's ability to run tasks in the background while the program is running interfered with the execution of the program, making the camera image buffer and freeze" \cite{katieshepherdMachineVisionBased2023}. With the analysis of RTOS in Section \ref{bi_rt}, I have decided that the project will be RT-driven by implementing the PREEMPT-RT kernel patch in Linux to minimise task switching latencies to mitigate the jitter issues as much as possible. Sacrificing performance for rapid development, Python enables real-time development for less stringent requirements.

\subsection{Optimisations Using Predictive Systems}
Although reducing the resolution of high-quality image frame captures is viable to reduce the computational overheads with machine vision applications, it often trades off object detection and segmentation accuracy. A predictive system approach can mitigate the computationally intensive requirement of machine vision application to every frame of an input video feed. Accurate backscatter particle movement and future location prediction eliminate the requirement to apply machine vision-based technologies to every frame in the video feed to identify and segment the particles. Initially, the project will implement a non-ML approach using interpolation with splines, which I must research in more detail as the project progresses. With interpolation in place, I must research ML-based classifier approaches with thought into generating the training data as the project progresses.
