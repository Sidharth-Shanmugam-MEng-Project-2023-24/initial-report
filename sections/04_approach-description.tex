The three project objectives form a waterfall-effect task cascade, powering the overarching software development process by providing major version milestones. Following the software version milestones, each development stint will consist of subtasks with a scope for agile methodologies. The prior completion of many project aspects, such as crucial background research, finalising and ordering parts, and some software development, makes room for much experimental work.

\subsection{Milestone 1: Basic Canny-Based Backscatter Segmentation (Software V1)}
With the technical background research in place, I have developed a software spike that uses the Canny edge detection algorithm. The spike utilises a GoPro footage of the sea floor. Due to the rolling-shutter sensor on board the GoPro and intense video compression, this footage has many artefacts and image distortion. The first step forward is to utilise the underwater testing facilities at the Institute for Safe Autonomy (ISA) to generate uncompressed footage of a stream of bubbles using the RPi global shutter camera system. So far, I have developed a Python script intended for the RPi computers with an installed RPi global shutter camera to record a video feed. Before recording, I must allocate sufficient time to update this script to record in raw format, as it currently uses H.264 encoding, and to set up the recording rig to use at the testing facilities. In parallel, I must research metrics for quantifying the real-time performance of the software to prepare for the V1 software requirements: (a) Canny-based backscatter segmentation, optimised with the test footage recording, and (b) real-time metrics tracking logic. There must be an excess buffer time to account for the paramount risk of insufficient optimisation time for the Canny algorithm and insufficient development time for modular software, otherwise resulting in a backlog of software development tasks, mainly for the real-time tracking metrics implementation. With the target for modular software development, future feature implementations will be seamless and will not require re-writing large code sections.

\subsection{Milestone 2: Advanced Canny-Based Backscatter Segmentation With Tracking (Software V2)}
Following the first milestone, the RPi test systems, consisting of the Pi 4 and the Pi 5, must contain the PREEMPT-RT kernel patch. The implementation must house simple logic for rapid switching between the PREEMPT-RT kernel and the standard kernel, aiding for quick parameterisation of performance. While this task should only take a day and has no paramount risks to cause overrun, as with all software, there must be a slight buffer. On the software side, the first requirement for V2 is the snake method implementation for a more accurate backscatter segmentation. In parallel, the second requirement is the research and development of the least distance rule and Kalman filter methodologies for backscatter tracking. In a fast-paced environment, such as underwater from a UUV, where the rapidly moving propellors disrupt backscatter particles at a snappy rate, it may be worth considering the computational performance and development time trade-offs with the Kalman filter-based tracking approach when the least distance approach may suffice, over-estimating the task duration to account for research time towards method evaluation to result in a single method deployment. The final requirement of this software version is the mitigation of parallax effects, in addition to implementing logic to drive the DLP projector, after which the real-time performance of backscatter cancellation between the standard and PREEMPT-RT kernel is ready for evaluation. Due to the nature of software development, there are many risks to consider for this milestone. Hence, a 1-week buffer is crucial before the milestone deadline.

\subsection{Deliverable 1: Demonstration of Software V2}
The completion of V2 from the previous milestone showcases a fully-fledged Canny edge detection-based backscatter cancellation system. A department-wide project demonstration to industry professionals on the 29th of April, 2024, forms the ideal platform to exhibit the project and to welcome feedback for improvement. Allocating some time to prepare for this presentation is vital, although over-estimating the time allocation to ensure there is sufficient buffer in case Milestone 2 is incomplete.

\subsection{Milestone 3: Predictive Backscatter Segmentation Approaches (Software V3)}
In parallel with the demonstration day preparation for milestone 2, there must be a focus on the third and final software iteration. Consisting of researching, implementing, and evaluating non-ML approaches, such as interpolative methodologies, with ML-based approaches, considering unsupervised and supervised learning methodologies, fulfils the requirements of the final software version. Research into layering the predictive logic on machine-vision technologies to enhance system efficiency by assessing real-time metrics will complete the third objective of this project. As with all software-related tasks, there must be a sufficient buffer in the outcome of unsuccessful deadline completion. It is increasingly paramount when incorporating ML technologies as not only is fine-tuning a requirement, but the training between each trial run can take a very long time. With the completion of V3, underwater testing can occur. Based on buffer consumption, not just limited to the underwater testing tank at the ISA, but with scope for real-life application with system deployment into a lake.

\subsection{Deliverables 2 \& 3: The Final Report, Project Presentation, and Viva Voce}
Only when the final testing runs and the acquisition of project evaluation material, such as real-time metrics, is complete can writing the final project report proceed. Due to the underestimated difficulty and time consumption of articulating metrics, results, and summaries, the plan must consider this task to be of monumental duration. The same applies to the final presentation and project viva. There must be sufficient buffer time on occasions where the pushback of a deadline is the only option, such as to mitigate an illness.
